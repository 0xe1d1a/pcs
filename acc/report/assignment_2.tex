%----------------------------------------------------------------------------------------
%	PACKAGES AND OTHER DOCUMENT CONFIGURATIONS
%----------------------------------------------------------------------------------------

\documentclass{article}

\usepackage{fancyhdr} % Required for custom headers
\usepackage{lastpage} % Required to determine the last page for the footer
\usepackage{extramarks} % Required for headers and footers
\usepackage[usenames,dvipsnames]{color} % Required for custom colors
\usepackage{graphicx} % Required to insert images
\usepackage{listings} % Required for insertion of code
\usepackage{courier} % Required for the courier font
\usepackage{lipsum} % Used for inserting dummy 'Lorem ipsum' text into the template
\usepackage{parskip}
% Margins
\topmargin=-0.45in
\evensidemargin=0in
\oddsidemargin=0in
\textwidth=6.5in
\textheight=9.0in
\headsep=0.25in

\linespread{1.1} % Line spacing

% Set up the header and footer
\pagestyle{fancy}
\chead{} % Top left header
\lhead{\hmwkClass\  \hmwkTitle} % Top center head
\rhead{} % Top right header
\lfoot{\lastxmark} % Bottom left footer
\cfoot{} % Bottom center footer
\rfoot{Page\ \thepage\ of\ \protect\pageref{LastPage}} % Bottom right footer
\renewcommand\headrulewidth{0.4pt} % Size of the header rule
\renewcommand\footrulewidth{0.4pt} % Size of the footer rule

\setlength\parindent{0pt} % Removes all indentation from paragraphs

%----------------------------------------------------------------------------------------
%	CODE INCLUSION CONFIGURATION
%----------------------------------------------------------------------------------------

% \definecolor{MyDarkGreen}{rgb}{0.0,0.4,0.0} % This is the color used for comments
\lstloadlanguages{C} % Load C syntax for listings, for a list of other languages supported see: ftp://ftp.tex.ac.uk/tex-archive/macros/latex/contrib/listings/listings.pdf
\lstset{language=C,frame=single,keywordstyle=[1]\color{Blue}\bf} % Use C in this example      



\newcommand{\csnippet}[2]{
\begin{itemize}
\item[]\lstinputlisting[caption=#2,label=#1]{#1.c}
\end{itemize}
}

%----------------------------------------------------------------------------------------
%	DOCUMENT STRUCTURE COMMANDS
%----------------------------------------------------------------------------------------

% Header and footer for when a page split occurs within a problem environment
\newcommand{\enterProblemHeader}[1]{
\nobreak\extramarks{#1}{#1 continued on next page\ldots}\nobreak
\nobreak\extramarks{#1 (continued)}{#1 continued on next page\ldots}\nobreak
}

% Header and footer for when a page split occurs between problem environments
\newcommand{\exitProblemHeader}[1]{
\nobreak\extramarks{#1 (continued)}{#1 continued on next page\ldots}\nobreak
\nobreak\extramarks{#1}{}\nobreak
}

\setcounter{secnumdepth}{0} % Removes default section numbers
\newcounter{homeworkProblemCounter} % Creates a counter to keep track of the number of problems

\newcommand{\homeworkProblemName}{}
\newenvironment{homeworkProblem}[1][Problem \arabic{homeworkProblemCounter}]{ % Makes a new environment called homeworkProblem which takes 1 argument (custom name) but the default is "Problem #"
\stepcounter{homeworkProblemCounter} % Increase counter for number of problems
\renewcommand{\homeworkProblemName}{#1} % Assign \homeworkProblemName the name of the problem
\section{\homeworkProblemName} % Make a section in the document with the custom problem count
\enterProblemHeader{\homeworkProblemName} % Header and footer within the environment
}{
\exitProblemHeader{\homeworkProblemName} % Header and footer after the environment
}

\newcommand{\problemAnswer}[1]{ % Defines the problem answer command with the content as the only argument
\noindent\framebox[\columnwidth][c]{\begin{minipage}{0.98\columnwidth}#1\end{minipage}} % Makes the box around the problem answer and puts the content inside
}

\newcommand{\homeworkSectionName}{}
\newenvironment{homeworkSection}[1]{ % New environment for sections within homework problems, takes 1 argument - the name of the section
\renewcommand{\homeworkSectionName}{#1} % Assign \homeworkSectionName to the name of the section from the environment argument
\subsection{\homeworkSectionName} % Make a subsection with the custom name of the subsection
\enterProblemHeader{\homeworkProblemName\ [\homeworkSectionName]} % Header and footer within the environment
}{
\enterProblemHeader{\homeworkProblemName} % Header and footer after the environment
}

%----------------------------------------------------------------------------------------
%	NAME AND CLASS SECTION
%----------------------------------------------------------------------------------------

\newcommand{\hmwkTitle}{Assignment\ \#4} % assignment title
\newcommand{\hmwkDueDate}{Monday,\ December\ 8,\ 2014} % due date
\newcommand{\hmwkClass}{Programming Concurrent Systems} % class
\newcommand{\hmwkClassTime}{} % lecture time
\newcommand{\hmwkClassInstructor}{} % lecturer
\newcommand{\hmwkAuthorName}{Alyssa - Ilias} %name

%----------------------------------------------------------------------------------------
%	TITLE PAGE
%----------------------------------------------------------------------------------------

\title{
\vspace{2in}
\textmd{\textbf{\hmwkClass:\ \hmwkTitle}}\\
\normalsize\vspace{0.1in}\small{Due\ on\ \hmwkDueDate}\\
\vspace{0.1in}\large{\textit{\hmwkClassInstructor\ \hmwkClassTime}}
\vspace{3in}
}

\author{\textbf{\hmwkAuthorName}}


%----------------------------------------------------------------------------------------

\begin{document}

\maketitle

%----------------------------------------------------------------------------------------
%	TABLE OF CONTENTS
%----------------------------------------------------------------------------------------

%\setcounter{tocdepth}{1} % Uncomment this line if you don't want subsections listed in the ToC

%\newpage
%\tableofcontents
\newpage

%----------------------------------------------------------------------------------------
%	Introduction
%----------------------------------------------------------------------------------------
\begin{homeworkProblem}[Introduction]

This assignment asked us to perform experiments using OpenACC. In this report, we present results from
the first two parts: first, experiments from accelerating our heat dissipation code, and then some
results from the provided matmul code.

\end{homeworkProblem}
%----------------------------------------------------------------------------------------
%	Heat
%----------------------------------------------------------------------------------------

% To have just one problem per page, simply put a \clearpage after each problem

\begin{homeworkProblem}[Heat dissipation | OpenACC]
\textbf{Solution description}

Our solution this time is based on the reference code, due to issues with the extensive
use of pointer arithmetic in our original code. We used the adapted version of the reference
code from Franz Geiger, which fixes compilation issues with the PGI compiler due to the lack
of support for C99 multidimensional variable-length arrays.

We rearranged the code to remove unnecessary smearing iterations from the computation loops,
and to move it all into a single function for simplicity.

We copy all three matrices (the conduction data, and the source/destination matrices) in at
the start of the main loop. The destination data is largely garbage at this point, and so a
small performance improvement could be obtained by only copying in the (smeared) edges, but
for simplicity we didn't do thisa.

\csnippet{main}{Main loop}

We parallelized the main dissipation computation, including the smearing. Note that we use
the \texttt{present} directive, which avoids an unnecessary copy but also makes the GPU
aware of the swapped destination/source pointers.

\csnippet{compute}{Dissipation computation}

And we also parallelized the reduction step, using OpenACC's directive:

\csnippet{reduction}{Reduction using acc directive}

(We also included OpenMP directives, but the resulting code is slower than our original OpenMP
code, so we didn't use it.)

\textbf{Evaluation - Experiments}

We run our experiments on the DAS-4 system. For pthreads/OpenMP, we used a normal node which has 8 physical cores.
For the OpenACC results, we used the nvidia (CUDA) backend of the PGI compiler, and ran the result on the DAS-4
systems with either a GTX480 or a Tesla C2050. The limited availability of the nodes with other GPU types made it
difficult to experiment with them.

\includegraphics[width=0.75\columnwidth]{effectivness.png}

The figure above shows the effectivness of the parallelisation. The experiments were made with the following parameters:

\begin{verbatim}
./heat -e 0.0 -i 2000 -k 2001
\end{verbatim}

\includegraphics[width=0.75\columnwidth]{effectivness.png}

Notice how, for the first time, effectiveness did not necessarily equate to small execution time, due to the communication overhead.

\includegraphics[width=0.75\columnwidth]{walltime.png} 

\includegraphics[width=0.75\columnwidth]{speedup_heat.png} 

A good GPU implementation of this problem would proceed using tiles/blocks (since the neighbours which are needed
by each computation are in all directions), ideally copying each block into shared memory rather than accessing global
memory repeatedly. Unfortunately, PGI doesn't support the OpenACC 2.0 \texttt{tile} pragma, and in any case, the
details of shared memory are not exposed.

We did, however, perform some experiments to try working out the optimal combination of gang, worker and vector
sizes to use. Unfortunately, these results were invalidated shortly before submitting this report when we realised
that we'd made a fundamental mistake in the loop iterations (we were iterating over the horizontal direction in the
outer loop, rather than the vertical one), but we present an example graph showing an example portion of our results
from these experiments anyway (PGI assigns the workers/vectors to dimensions of \texttt{threadIdx}, so the symmetric
nature of the graph is to be expected):

\includegraphics[width=0.75\columnwidth]{heat.png}

\includegraphics[width=0.75\columnwidth]{effectivness_withreductions.png} 

\includegraphics[width=0.75\columnwidth]{walltime_withreductions.png}

\includegraphics[width=0.75\columnwidth]{speedup_withreductions.png} 

\end{homeworkProblem}

%----------------------------------------------------------------------------------------
%	Merge
%----------------------------------------------------------------------------------------

\begin{homeworkProblem}[Matmul]

Small intro here

\includegraphics[width=0.75\columnwidth]{fig.png}

\includegraphics[width=0.75\columnwidth]{speedup.png} 

\includegraphics[width=0.75\columnwidth]{bigfig.png} 

\includegraphics[width=0.75\columnwidth]{matmul_more_elements.png}

%one more fig coming here just like fig1 but with more elements	

\end{homeworkProblem}

%----------------------------------------------------------------------------------------

\end{document}
